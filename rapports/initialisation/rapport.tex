\documentclass[a4paper, 12pt, french]{article}
\usepackage[utf8]{inputenc}
\usepackage[french]{babel}
\usepackage[T1]{fontenc}
\usepackage[babel=true]{csquotes}
\usepackage{palatino}
\usepackage{hyperref}
\usepackage{eurosym}
\usepackage{graphicx}
\usepackage{array}
\usepackage{fancyhdr}
\usepackage{lastpage}
\usepackage{xspace}

%% -- Commandes personalisées

% Contrainte : chaîne de caractères sans espace, en minuscule, constituée seulement des 26 caractères non accentués de l’alphabet latin
\newcommand{\nomProjet}{Nom du projet\xspace}
% Contrainte : chaîne de caractères sans espace, en minuscule, constituée seulement des 26 caractères non accentués de l’alphabet latin
\newcommand{\nomEquipe}{Nom de l'équipe\xspace}

\pagestyle{fancy}
% En-têtes
\renewcommand{\headrulewidth}{1pt}
\fancyhead[C]{\nomProjet} 
\fancyhead[L]{Fiche d'initialisation}
\fancyhead[R]{machin}

\renewcommand{\footrulewidth}{1pt}
\fancyfoot[C]{Version 1.0} 
\fancyfoot[L]{\nomEquipe\xspace G1}
\fancyfoot[R]{Page \thepage\xspace sur \pageref{LastPage}}

%% -- Document
\begin{document}
	\begin{titlepage}
		\begin{center}
			\LARGE{\bsc{Fiche d'initialisation}} \\
		    \rule{\linewidth}{1.5pt}
		    \huge{\textbf{\nomProjet}}
		    \rule{\linewidth}{1.5pt} \newline{} \newline{}
		\end{center}
		\begin{center}
		    \large{Auteurs :}\\ Antoine \bsc{Augusti} (ANT)\\ Étienne \bsc{Batise} (BAT) \\ Thibaud \bsc{Dauce} (TIB)
		\end{center}
		\vspace{50px}
		\begin{center}
			\large{Date de rédaction :}\\ \today
		\end{center}
	\end{titlepage}
	\newpage

	% Table des matières
	\tableofcontents
	\pagebreak

	% Début du document
	\section{Nom de l'équipe}
	Notre équipe se nommera \nomEquipe.

	\section{Liste des membres de l'équipe}
	Notre équipe sera composée des personnes suivantes :
	\begin{itemize}
		\item Antoine Augusti
		\item Étienne Batise
		\item Thibaud Dauce 
	\end{itemize}

	\section{Nom du service}
	Le service que nous souhaitons réaliser dans le cadre de ce cours se nomme \nomProjet.

	\section{Notre concept}
	\nomProjet est une application qui permet de gérer les dettes entre colocataires.
\end{document}