\documentclass[a4paper, 12pt, french]{report}
\usepackage[utf8]{inputenc}
\usepackage[french]{babel}
\usepackage[T1]{fontenc}
\usepackage[babel=true]{csquotes}
\usepackage{palatino}
\usepackage{hyperref}
\usepackage{eurosym}
\usepackage{graphicx}
\usepackage{array}
\usepackage{fancyhdr}
\usepackage{lastpage}
\usepackage{xspace}
\usepackage{geometry}
\addtolength{\oddsidemargin}{-.875in}
\addtolength{\evensidemargin}{-.875in}
\addtolength{\textwidth}{1.75in}

\addtolength{\topmargin}{-.5in}
\addtolength{\textheight}{1.75in}
%% -- Commandes personalisées

% Contrainte : chaîne de caractères sans espace, en minuscule, constituée seulement des 26 caractères non accentués de l’alphabet latin
\newcommand{\nomProjet}{enflatme\xspace}
% Contrainte : chaîne de caractères sans espace, en minuscule, constituée seulement des 26 caractères non accentués de l’alphabet latin
\newcommand{\nomEquipe}{teamflat\xspace}

\pagestyle{fancy}
% En-têtes
\renewcommand{\headrulewidth}{1pt}
\fancyhead[C]{\nomProjet} 
\fancyhead[L]{Cahier des charges}
\fancyhead[R]{}

\renewcommand{\footrulewidth}{1pt}
\fancyfoot[C]{Version 1.0} 
\fancyfoot[L]{\nomEquipe\xspace G1}
\fancyfoot[R]{Page \thepage\xspace sur \pageref{LastPage}}

%% -- Document
\begin{document}
	\begin{titlepage}
		\begin{center}
			\LARGE{\bsc{Cahier des charges}} \\
		    \rule{\linewidth}{1.5pt}
		    \huge{\textbf{\nomProjet}}
		    \rule{\linewidth}{1.5pt} \newline{} \newline{}
		\end{center}
		\begin{center}
		    \large{Auteurs :}\\ Antoine \bsc{Augusti} (ANT)\\ Étienne \bsc{Batise} (BAT) \\ Thibaud \bsc{Dauce} (TIB)
		\end{center}
		\vspace{50px}
		\begin{center}
			\large{Date de publication :}\\ \today
		\end{center}
	\end{titlepage}
	\newpage

	% Table des matières
	\tableofcontents
	\pagebreak

	% Début du document
	\chapter{Fondements du projet}
		\section{But du projet}
			\subsection{Problème de l'utilisateur}
Notre objectif est de proposer une application pour simplifier la vie des gens vivant dans une colocation. La vie entre colocataires implique des contraintes et des problèmes ; l'objectif de notre application est de les gérer plus facilement et plus efficacement.

\subsection{Objectifs du projet}
Nous voulons réaliser cette application pour qu'elle s'impose comme un outil incontournable de gestion pour toutes les personnes vivant dans une colocation. Nous voulons leur rendre la vie plus facile en faisant en sorte que les petits tracas de la vie entre colocataires soient fortement réduits.

		\section{Personnes et organismes impliqués dans les enjeux du projet}
			\subsection{Maître d'ouvrage}
Les maîtres d'ouvrage sont :
\begin{itemize}
	\item Antoine Augusti, étudiant à l'Institut National des Sciences Appliquées de Rouen ; 
	\item Étienne Batise, étudiant à l'Institut National des Sciences Appliquées de Rouen ; 
	\item Thibaud Dauce, étudiant à l'Institut National des Sciences Appliquées de Rouen.
\end{itemize}

\subsection{Acheteur}
L'acheteur est la société Teamflat où travaillent les maîtres d'ouvrage cités précédemment.

		\section{Utilisateurs du produit}
			\subsection{Utilisateurs directs du produit}
Les utilisateurs de notre application seront les personnes vivant dans une colocation. La majorité des personnes vivant dans une colocation sont des étudiants puisque ceux-ci ont souvent peu de moyens financiers et choisissent de vivre en colocation pour réaliser des économies.\\

Notre cible est donc un public jeune, entre 18 et 25 ans, autant hommes que femmes. Certains jeunes actifs font également le choix de la colocation. Notre public utilise très fréquemment et maîtrise les dernières nouveautés technologiques et télécharge très régulièrement de nouvelles applications.\\

Concernant les centres d'intérêt de notre cible ils sont autour des sorties, de la musique, des soirées, des jeux-vidéos, des sports\dots Notre cible n'a pas beaucoup de temps à consacrer à la réalisation de tâches ménagères et n'apprécie pas perdre du temps avec ceci. 

\subsection{Priorité assignée aux utilisateurs}
Les utilisateurs sont notre priorité ultime car nous réalisons une application pour répondre à leurs besoins. De plus, nous comptons sur nos utilisateurs pour réaliser la promotion de notre application et ainsi faire en sorte que celle-ci ait une croissance virale.

\subsection{Implication nécessaire de la part des utilisateurs dans le projet}
Aucune implication particulière des utilisateurs n'est nécessaire pour mener à bien le projet. Toutefois, étant donné qu'il est primordial de concevoir un produit qui convient à notre cible que sont les colocataires, il est important de travailler de manière proche avec certains d'entre eux afin d'avoir leur avis sur l'avancement et l'orientation du projet.

\subsection{Utilisateurs concernés par les opérations de maintenance du produit}
Aucun utilisateur ne devra être concerné par des opérations de maintenance car les données devront être stockée sur un des serveurs de la société Teamflat.

	\chapter{Contraintes sur le projet}
	\chapter{Exigences fonctionnelles}
		\section{Portée du travail}
			\subsection{La situation actuelle}
				Aujourd'hui, il n'existe aucun type d'application correspondant à la description de notre produit. Cependant, il existe une application web publique développé par Merlin Nimier-David qui permet, de manières simpliste, à différent membres d'un groupe de gérer leurs dettes. Cette application est accessible à l'adresse \url{merlin.nimierdavid.fr/debts/}.\\

Par ailleurs, il existe aussi de nombreuse applications web / mobile permettant de créer ses listes de courses. L'une des plus connues sur le marché est l'application ``PlanCourses'' développée par la société Agilys qui permet notamment de choisir choisir ses articles en fonction des marques de la grande distribution comme Carrefour, Auchan, Leclerc\ldots

			\subsection{Contexte du travail}
			\subsection{Division du travail en événement métier}
				\paragraph{Permettre le changement de colocation\\}
Il est possible dans la vie d'un utilisateur de déménager, il est donc nécessaire de lui permettre de changer ou quitter sa colocation pour, s'il le souhaite, en rejoindre une autre.

\paragraph{Être agréable à utiliser\\}
Pour que notre produit ait un réel succès, il est nécessaire que l'utilisateur n'apprécie pas le produit uniquement pour le service  qu'il propose. Il faut qu'il apprécie et aime l'utiliser ; il s'agit donc de s'inspirer des nombreux services populaires actuellement comme les réseaux sociaux.

\paragraph{Afficher clairement la gestion des tâches\\}
Notre produit est destiné à rendre plus facile la gestion d'une colocation et à servir de support pour prévenir les problèmes. Ainsi, il faut permettre aux utilisateurs d'interpréter facilement et de manière compréhensible leurs propres données.
		\section{Portée du Produit}
			\subsection{Modèle d'usage}
			\subsection{Description des cas d'utilisation}

		\section{Portée fonctionnelles}

	\chapter{Exigences non fonctionnelles}
		\section{Ergonomie et convivialité du produit}
			\subsection{L'interface}
L'application utilisera un design dit \textit{flat}, c'est-à-dire à posséder une interface avec des formes simples et nettes (pas de bords arrondis), sans ombres et sans dégradés. L'interface devra être uniforme entre l'application web et l'application mobile Android. L'application devra posséder une mascotte, qui reste encore à définir, qui aura vocation d'accompagner l'utilisateur dans la prise en main de l'application. Celle-ci sera présente et possédera une personnalité bien définie.\\

Les interactions avec l'interface doivent être rapides : l'interface doit être réactive et le nombre de gestes pour réaliser des actions doit être minimal.

\subsection{Le style du produit}
Le public doit correspondre aux interfaces que les jeunes ont l'habitude d'utiliser en ce moment, on pense particulièrement aux réseaux sociaux. L'application devra être colorée et être perçue comme moderne par nos utilisateurs.

		\section{Facilité d'utilisation et facteurs humains}
			\subsection{Facilité d'utilisation}
La prise en main de l'application doit se faire rapidement. C'est pourquoi on utilisera dès que possible des petits didacticiels pour montrer comment l'interface doit être utilisée. Pour ceci on pourra s'aider de notre mascotte où d'écrans de présentation. L'utilisateur doit pouvoir mémoriser facilement l'interface, c'est pourquoi on essaiera d'uniformiser celle-ci au maximum entre les différentes fonctionnalités.

\subsection{Personnalisation et internationalisation}
Notre application sera utilisable en Français et utilisera l'Euro comme monnaie. Il n'est pas prévu de proposer d'autres langues et d'autres monnaies dans un premier temps.

\subsection{Facilité d'apprentissage}
L'apprentissage de l'utilisation de l'application doit être immédiat pour éviter de perdre des utilisateurs rebutés par une interface trop complexe à comprendre.

\subsection{Facilité de compréhension et politesse}
L'application utilisera beaucoup d'icônes et aura un ton proche de l'utilisateur qui correspond à la culture des jeunes. Les messages pourront être ironiques et taquins dans des situations non critiques.

\subsection{Exigences d'accessibilité}
Aucune exigence particulière concernant les handicaps. Il est nécessaire de s'assurer que les éléments de l'interface ne soient pas trop petits pour pouvoir être touchés sur un mobile sans risque d'erreur.

		\section{Fonctionnement du produit}
			\subsection{Rapidité d'exécution et temps de latence}
La transition entre les écrans devra être la plus rapide possible. Pour l'application mobile, lorsque l'utilisateur n'est pas connecté au réseau Internet on mettra en cache les données qu'il souhaite envoyer et consulter. Celles-ci seront alors récupérées la prochaine fois qu'il sera en ligne.

\subsection{Exigences critiques de sécurité}
L'utilisation du produit n'entrainera pas un risque de sécurité particulier.

		\section{Adéquation du produit avec son environnement}
			\subsection{Environnement physique prévu}
Le site web sera utilisé dans un lieu calme, sur un ordinateur fixe ou portable où l'utilisateur pourra facilement se concentrer sur ce qu'il a à accomplir. Les utilisations de l'application mobile doivent pouvoir s'effectuer rapidement et ont pour vocation d'être courtes. Il est préférable d'inciter l'utilisateur à effectuer des tâches longues ou non adaptées pour le mobile sur le site web. On privilégiera l'application mobile pour des vérifications ou de légères mises à jour de statuts.

\subsection{Environnement technologique prévu}
Le site web doit être compatible avec tous les navigateurs (y compris Internet Explorer) pour les versions sorties il y a moins de 2 ans. L'application mobile doit être utilisable sur tous les téléphones Android pour une version supérieure à 4.

\subsection{Applications partenaires}
Nous ne souhaitons pas d'interactions avec des plate-formes extérieures, en particulier les réseaux sociaux.
\end{document}