\subsection{Facilité d'utilisation}
La prise en main de l'application doit se faire rapidement. C'est pourquoi on utilisera dès que possible des petits didacticiels pour montrer comment l'interface doit être utilisée. Pour ceci on pourra s'aider de notre mascotte où d'écrans de présentation. L'utilisateur doit pouvoir mémoriser facilement l'interface, c'est pourquoi on essaiera d'uniformiser celle-ci au maximum entre les différentes fonctionnalités.

\subsection{Personnalisation et internationalisation}
Notre application sera utilisable en Français et utilisera l'Euro comme monnaie. Il n'est pas prévu de proposer d'autres langues et d'autres monnaies dans un premier temps.

\subsection{Facilité d'apprentissage}
L'apprentissage de l'utilisation de l'application doit être immédiat pour éviter de perdre des utilisateurs rebutés par une interface trop complexe à comprendre.

\subsection{Facilité de compréhension et politesse}
L'application utilisera beaucoup d'icônes et aura un ton proche de l'utilisateur qui correspond à la culture des jeunes. Les messages pourront être ironiques et taquins dans des situations non critiques.

\subsection{Exigences d'accessibilité}
Aucune exigence particulière concernant les handicaps. Il est nécessaire de s'assurer que les éléments de l'interface ne soient pas trop petits pour pouvoir être touchés sur un mobile sans risque d'erreur.