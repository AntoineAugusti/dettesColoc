\subsection{Rapidité d'exécution et temps de latence}
La transition entre les écrans devra être la plus rapide possible. Pour l'application mobile, lorsque l'utilisateur n'est pas connecté au réseau Internet on mettra en cache les données qu'il souhaite envoyer et consulter. Celles-ci seront alors récupérées la prochaine fois qu'il sera en ligne.

\subsection{Exigences critiques de sécurité}
L'utilisation du produit n’entraînera pas un risque de sécurité particulier.

\subsection{Précision et exactitude}
Les heures seront données au maximum avec une exactitude ne dépassant pas la minute. Les unités monétaires utiliseront une précision de l'ordre du centième d'euro.

\subsection{Fiabilité et disponibilité}
Les serveurs hébergeant le site web et l'application du projet devront avoir une disponibilité de 99,5 \% (1,83 jours d'indisponibilité par an).

\subsection{Robustesse ou tolérance à un emploi erroné}
L'application mobile devra fonctionnement de manière correcte si celle-ci ne possède pas d'accès à Internet. Certaines fonctionnalités pourront être limitées pour ne pas avoir un comportement inhabituel ou des données incohérentes.

\subsection{Capacité de stockage et montée en charge}
Le produit doit pouvoir servir simultanément à 500 utilisateurs entre 16h et 23h (heures de pointe). La charge maximale aux autres moments sera de 200 utilisateurs.

\subsection{Adaptation du produit à une augmentation de volume à traiter}
Le produit doit pouvoir gérer le passage d'une base vide à une base de 100 000 utilisateurs en l'espace d'un mois.

\subsection{Longévité}
Le produit n'a pas de durée de vie définie. Il doit s'adapter aux évolutions pour pouvoir durer le plus longtemps possible. On estime la durée de vie d'utilisation de l'application par un utilisateur à 2-3 ans en moyenne, soit le temps moyen où celui-ci est en colocation.
Le produit n'a pas de durée de vie définie. Il doit s'adapter aux évolutions pour pouvoir durer le plus longtemps possible. On estime la durée de vie d'utilisation de l'application par un utilisateur à 2-3 ans en moyenne, soit le temps moyen où celui-ci est en colocation.