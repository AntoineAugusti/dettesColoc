\paragraph{Collocation}
La collocation désigne l'habitation des collocataires. Celle-ci peut être un appartement, une maison ou tout autre logement.

\paragraph{Smartphone}
Un smartphone est un téléphone mobile équipé d'un système d'exploitation permettant d'accèder à Internet et d'effectuer des tâches plus complexes que de téléphoner ou envoyer des SMS. Nous entendons par smartphone, tout téléphone équipé du système d'exploitation Android, iOS, Windows Phone, Sailfish OS ou Firefox OS.

\paragraph{License libre}
Les licenses libres permettent de mettre à disposition de la communauté le code source d'un logiciel ou d'un programme afin qu'il puisse être réutilisé, modifié et amélioré. Dans ce document nous parlons d'une license de type copyleft qui aurait pour objectif de bénéficier des améliorations qui pourraient être apportées à notre logiciel par la communauté.

\paragraph{Camel case}
Le camel case est une norme d'écriture facilitant la lecture. Un mot en camel case est écrit en miniscule. Dans le cas d'un groupe de mot attachés, la première lettre est une minuscule, chaque nouveau mot commence par une majsucule.\\
Exemple : variable trop cool$ \Rightarrow $variableTropCool

\paragraph{Conditions ternaires}
Certains langages proposent une écriture simplifiée pour les conditions de type 'if / else' appelée condition ternaire et tenant sur une seule ligne.

\paragraph{Application native}
Une application native est une application installable par l'utilisateur sur son appareil. Elle est développée dans un langage particulier, souvent propre à la plate-forme ou possèdant des bibliothèques propres à la plate-forme. Elle présente l'avantage d'être rapide et permet l'utilisation de manière très facile des composants particuliers des appareils (géolocalisation, vibreur...).

\paragraph{Application web}
Une application web est une application manipulable à travers un navigateur web. Elle présente l'avantage d'être utilisable sur toutes les plate-formes (Android, iOS, Windows, Linux...) sans installation de la part de l'utilisateur. La démocratisation du HTML5 permet aux applications web d'accèder à des composants particuliers des appareils (géolocalisation, vibreur...)

