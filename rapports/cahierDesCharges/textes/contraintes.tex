\subsection{Contraintes sur la conception de la solution} % (fold)
\label{sub:contraintes_sur_la_conception_de_la_solution}

	\subsubsection{Téléphone mobile Android}
	L'application doit être utilisable sur un smartphone Android équipé d'une version du système d'exploitation supérieure à la version 4 afin de cibler le maximum d'utilisateurs.

	\subsubsection{Code source}
	Le code source doit répondre à certaines exigences afin d'être libéré sous licence libre dans un second temps :\\

	Le code source doit être lisible. Les développeurs ne doivent pas utiliser d’abréviations pour les noms des variables, le camelCase doit être privilégié pour le nom de variables et des fonctions. Les développeurs doivent dans la mesure du possible utiliser des opérateurs littéraux (comme ``AND'' ou ``OR'') à la place d'opérateurs moins lisibles (comme ``\&\&'' ou ``||''). Les développeurs ne doivent en aucun cas utiliser des conditions ternaires. Pour finir, le langage de programmation choisi doit permettre une lecture facile.\\

	Le code source doit être commenté de manière efficace. Les commentaires ne doivent pas surcharger le code source mais doivent être présents afin d'expliquer les parties complexes des fonctions. La décomposition d'un problème doit toujours être privilégiée à la surcharge d'une fonction. Les fonctions et les parties de code dites ``simples'' et compréhensibles par tous ne doivent pas être commentées. Un code lisible doit toujours être privilégié par rapport à des commentaires.\\

	Enfin, les fonctions du code source doivent être documentées afin de pouvoir générer une documentation de manière automatique (avec des outils tels que Doxygen ou JavaDoc par exemple). Chaque fonction doit présenter une description succincte, le nom des paramètres en entrée et le nom des paramètres en sortie. En cas de complexité plus importante ou d'une incompréhension possible, la documentation doit afficher une description plus complète et/ou une description de chaque attribut en entrée ou en sortie.


% subsection contraintes_sur_la_conception_de_la_solution (end)

\subsection{Environnement de fonctionnement du système actuel} % (fold)
\label{sub:environnement_de_fonctionnement_du_syst_me_actuel}

	Le but du projet est de créer une application à partir de zéro. Aucun système actuel n'est en place.

% subsection environnement_de_fonctionnement_du_syst_me_actuel (end)

\subsection{Applications partenaires} % (fold)
\label{sub:applications_partenaires}

	Nous n'envisageons aucun lien avec des applications tierces. En particulier avec les réseaux sociaux, notre application doit être complètement indépendante.

% subsection applications_partenaires (end)

\subsection{COTS : progiciels ou composants commerciaux} % (fold)
\label{sub:cots_progiciels_ou_composants_commerciaux}

	N/A

% subsection cots_progiciels_ou_composants_commerciaux (end)

\subsection{Lieux de fonctionnement prévus} % (fold)
\label{sub:lieux_de_fonctionnement_pr_vus}

	Le lieu principal de fonctionnement prévu est la collocation. Les personnes seront dans un environnement familier, sans bruit et tranquille. Ils auront à disposition un ordinateur fixe ou portable ainsi qu'un smartphone.

	Malgré la tranquillitéé du lieu, l'application doit être rapidement exécutable afin d'effectuer les mises à jour de ses activités rapidement.

% subsection lieux_de_fonctionnement_pr_vus (end)

\subsection{De combien de temps les développeurs disposent-ils pour le projet} % (fold)
\label{sub:de_combien_de_temps_les_d_veloppeurs_disposent_ils_pour_le_projet}

	Le projet doit être opérationnel fin août afin de permettre aux nouveaux étudiants de s'inscrire dès le début de leur collocation. Les développeurs disposent donc de 4 mois à compter d'aujourd'hui afin de livrer l'application début août et ainsi nous permettre de mettre en place une communication et une campagne marketing pour le lancement du produit.

% subsection de_combien_de_temps_les_d_veloppeurs_disposent_ils_pour_le_projet (end)

\subsection{Quel est le budget affecté au projet ?} % (fold)
\label{sub:quel_est_le_budget_affect_au_projet}

	N/A
	
% subsection quel_est_le_budget_affect_au_projet (end)

