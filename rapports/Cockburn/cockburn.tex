\documentclass[a4paper, 12pt, french]{article}
\usepackage[utf8]{inputenc}
\usepackage[french]{babel}
\usepackage[T1]{fontenc}
\usepackage[babel=true]{csquotes}
\usepackage{palatino}
\usepackage{hyperref}
\usepackage{eurosym}
\usepackage{graphicx}
\usepackage{array}
\usepackage{fancyhdr}
\usepackage{lastpage}
\usepackage{xspace}
\usepackage{longtable}
\usepackage{float}
\usepackage{geometry}
\addtolength{\oddsidemargin}{-.875in}
\addtolength{\evensidemargin}{-.875in}
\addtolength{\textwidth}{1.75in}

\addtolength{\topmargin}{-.5in}
\addtolength{\textheight}{1.75in}
%% -- Commandes personalisées

% Contrainte : chaîne de caractères sans espace, en minuscule, constituée seulement des 26 caractères non accentués de l’alphabet latin
\newcommand{\nomProjet}{enflatme\xspace}
% Contrainte : chaîne de caractères sans espace, en minuscule, constituée seulement des 26 caractères non accentués de l’alphabet latin
\newcommand{\nomEquipe}{teamflat\xspace}

\pagestyle{fancy}
% En-têtes
\renewcommand{\headrulewidth}{1pt}
\fancyhead[C]{\nomProjet} 
\fancyhead[L]{Fiche de Cockburn}
\fancyhead[R]{Cas d'utilisation 3}

\renewcommand{\footrulewidth}{1pt}
\fancyfoot[C]{Version 1.0} 
\fancyfoot[L]{\nomEquipe\xspace G1}
\fancyfoot[R]{Page \thepage\xspace sur \pageref{LastPage}}

\newcommand{\retourLigne}[2][c]{\begin{tabular}[#1]{@{}l@{}}#2\end{tabular}}

%% -- Document
\begin{document}
	\begin{titlepage}
		\begin{center}
			\LARGE{\bsc{Liste des cas d'utilisation}} \\
		    \rule{\linewidth}{1.5pt}
		    \huge{\textbf{\nomProjet}}
		    \rule{\linewidth}{1.5pt} \newline{} \newline{}
		\end{center}
		\begin{center}
		    \large{Auteurs :}\\ Antoine \bsc{Augusti} (ANT)\\ Étienne \bsc{Batise} (BAT) \\ Thibaud \bsc{Dauce} (TIB)
		\end{center}
		\vspace{50px}
		\begin{center}
			\large{Date de publication :}\\ \today
		\end{center}
	\end{titlepage}
	\newpage


\section{Titre}
Se connecter à l'application

\section{Résumé }
Liste des actions pour la connexion d'un utilisateur à l'application 

\section{Acteurs}
* Utilisateur (U)

\section{Pré-conditions}

\begin{itemize}
	\item Le smart-phone est allumé
	\item Le smart-phone possède l'application
	\item Le compte de U a déjà été créé par l'administrateur du système. *Voir cas 1*
	\item U a déjà rédifini son mot de passe.
\end{itemize}

\section{Action départ}
\begin{itemize}
	\item U souhaite se connecter à l'application
\end{itemize}

\section{Liste des actions}
	\begin{longtable}{|l|l|}
 	\hline
 	\textbf{Action de l'utilisateur} & \textbf{Action du système} \\ \hline
 	\hline
	\retourLigne{1. U lance l'application} &   \\ \hline
	& \retourLigne{2. Le système (S) demande le mot de passe à l'utilisateur} \\ \hline
	\retourLigne{3. U saisi son mot de passe } & \retourLigne{4. S vérifie la validité du mot de passe} \\ \hline
	& \retourLigne{5. S signale si le mot de passe n'est pas valide}  \\ \hline
	& \retourLigne{6. S affiche la page d'accueil} \\
	\hline
	\end{longtable}

\section{Action d'arrêt}

U a réussi à accéder à la page d'accueil du système

\section{Post-conditions}

\begin{itemize}
	\item L'application est en maintenance (inutilisable temporairement) 
	\item Le téléphone n'a plus de batterie.
\end{itemize}

\section{Exceptions}
\textit{Aucune exceptions trouvées}


\end{document}