\documentclass[a4paper, 12pt, french]{article}
\usepackage[utf8]{inputenc}
\usepackage[french]{babel}
\usepackage[T1]{fontenc}
\usepackage[babel=true]{csquotes}
\usepackage{palatino}
\usepackage{hyperref}
\usepackage{eurosym}
\usepackage{graphicx}
\usepackage{array}
\usepackage{fancyhdr}
\usepackage{lastpage}
\usepackage{xspace}
\usepackage{longtable}
\usepackage{array}
\usepackage{pifont}
\usepackage{amssymb}
\usepackage{geometry}
\addtolength{\oddsidemargin}{-.875in}
\addtolength{\evensidemargin}{-.875in}
\addtolength{\textwidth}{1.75in}

\addtolength{\topmargin}{-.5in}
\addtolength{\leftmargin}{-10in}
\addtolength{\textheight}{1.75in}

%% -- Commandes personalisées

% Contrainte : chaîne de caractères sans espace, en minuscule, constituée seulement des 26 caractères non accentués de l’alphabet latin
\newcommand{\nomProjet}{enflatme\xspace}
% Contrainte : chaîne de caractères sans espace, en minuscule, constituée seulement des 26 caractères non accentués de l’alphabet latin
\newcommand{\nomEquipe}{teamflat\xspace}
% Le type de document rédigé
\newcommand{\nomDocument}{Éxigences qualifiées annexe}

\pagestyle{fancy}
% En-têtes
\renewcommand{\headrulewidth}{1pt}
\fancyhead[C]{\nomProjet} 
\fancyhead[L]{\nomDocument}
\fancyhead[R]{}

\renewcommand{\footrulewidth}{1pt}
\fancyfoot[C]{Version 1.0} 
\fancyfoot[L]{\nomEquipe\xspace G1}
\fancyfoot[R]{Page \thepage\xspace sur \pageref{LastPage}}

%% -- Document
\begin{document}

	\section{Exigence 1 : opérationnelle en septembre}

	\paragraph{Critère de satisfaction}
	Afin de disposer du plus large pannel d'étudiants, l'application doit sortir avant le 15 septembre.

	\paragraph{Contentement du MAO}
	2

	\paragraph{Mécontentement du MAO}
	4

	\section{Exigence 3 : permettre d'envoyer des alertes}

	\paragraph{Critère de satisfaction}
	Afin que l'alerte soit efficace, elle doit pouvoir afficher une notification sur le téléphone en cas d'application native. En cas d'application Web, un email doit être envoyé à l'utilisateur pour le prévenir.

	\paragraph{Contentement du MAO}
	4

	\paragraph{Mécontentement du MAO}
	3

	\section{Exigence 11 : utilisable sur smartphone}

	\paragraph{Critère de satisfaction}
	L'application doit être utilisable sur Android de version 4 ou supérieur au minimum afin de satisfaire cette exigence. En cas d'application Web, le navigateur Chrome mobile et Firefox mobile doivent être complètement compatible avec leur dernière version. En cas d'application native, la plateforme Android est la seule requise. Une application native iOS et Windows 8 est un plus.

	\paragraph{Contentement du MAO}
	2

	\paragraph{Mécontentement du MAO}
	4

	\section{Exigence 13 : utilisable hors ligne}

	\paragraph{Critère de satisfaction}
	L'application devrait, au minimum, pouvoir afficher les données de la coloc hors ligne. Au mieux, l'utilisateur peut mettre à jour des informations hors ligne qui seront propoagées sur le serveur dès que le téléphone trouve un réseau.

	\paragraph{Contentement du MAO}
	4

	\paragraph{Mécontentement du MAO}
	2

	\section{Exigence 14 : limiter les téléchargements}

	\paragraph{Critère de satisfaction}
	Si l'application est une page Web, le poids de la page ne doit pas dépasser les 200ko. Dans le cas d'une application native, les données téléchargées à chaque chargement ne doivent pas exceder 100ko.

	\paragraph{Contentement du MAO}
	5

	\paragraph{Mécontentement du MAO}
	4

\end{document}