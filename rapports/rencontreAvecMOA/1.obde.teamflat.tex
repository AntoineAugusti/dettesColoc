\documentclass[a4paper, 12pt, french]{article}
\usepackage[utf8]{inputenc}
\usepackage[french]{babel}
\usepackage[T1]{fontenc}
\usepackage[babel=true]{csquotes}
\usepackage{palatino}
\usepackage{hyperref}
\usepackage{eurosym}
\usepackage{graphicx}
\usepackage{array}
\usepackage{fancyhdr}
\usepackage{lastpage}
\usepackage{xspace}
\usepackage{longtable}
\usepackage{float}
\usepackage{geometry}
\addtolength{\oddsidemargin}{-.875in}
\addtolength{\evensidemargin}{-.875in}
\addtolength{\textwidth}{1.75in}

\addtolength{\topmargin}{-.5in}
\addtolength{\textheight}{1.75in}
%% -- Commandes personalisées

% Contrainte : chaîne de caractères sans espace, en minuscule, constituée seulement des 26 caractères non accentués de l’alphabet latin
\newcommand{\nomProjet}{enflatme\xspace}
% Contrainte : chaîne de caractères sans espace, en minuscule, constituée seulement des 26 caractères non accentués de l’alphabet latin
\newcommand{\nomEquipe}{teamflat\xspace}

\pagestyle{fancy}
% En-têtes
\renewcommand{\headrulewidth}{1pt}
\fancyhead[C]{\nomProjet} 
\fancyhead[L]{Compte-rendu de la réunion avec la MOA}
\fancyhead[R]{}

\renewcommand{\footrulewidth}{1pt}
\fancyfoot[C]{Version 1.0} 
\fancyfoot[L]{\nomEquipe\xspace G1}
\fancyfoot[R]{Page \thepage\xspace sur \pageref{LastPage}}

\newcommand{\retourLigne}[2][c]{\begin{tabular}[#1]{@{}l@{}}#2\end{tabular}}

%% -- Document
\begin{document}
	\begin{titlepage}
		\begin{center}
			\LARGE{\bsc{Compte-rendu de la réunion avec la MOA}} \\
		    \rule{\linewidth}{1.5pt}
		    \huge{\textbf{\nomProjet}}
		    \rule{\linewidth}{1.5pt} \newline{} \newline{}
		\end{center}
		\begin{center}
		    \large{Auteurs :}\\ Antoine \bsc{Augusti} (ANT)\\ Étienne \bsc{Batise} (BAT) \\ Thibaud \bsc{Dauce} (TIB)
		\end{center}
		\vspace{50px}
		\begin{center}
			\large{Date de publication :}\\ \today
		\end{center}
	\end{titlepage}
	\newpage

	\section{Cartouche}
		\textbf{Objet de la réunion} : Lancement du projet obde de l'équipe bidideco le 28 mars 2014.
		\paragraph{Participants}{
			\begin{itemize}
				\item Antoine Augusti, MOE, présent ;
				\item Étienne Batise, MOE, présent ;
				\item Mickaël Billiotte, MOA, présent ;
				\item Anthony Courtin, MOA, présent ;
				\item Thibaud Dauce, MOE, présent ;
				\item Faustine Demiselle, MOA, présent ;
				\item Quentin Diaferia, MOA, présent.
			\end{itemize}
		}
		\vspace{20px}
		Le compte-rendu de la réunion sera conversé par l'équipe MOE teamflat.
	\section{Ordre du jour}
		L'ordre du jour est le lancement du projet obde de l'équipe bidideco, maîtrise d’œuvre de ce projet. Cette réunion a pour but la présentation du projet obde par l'équipe bidideco. Cette présentation doit être suivie d'une séance de questions-réponses pour clarifier les points du cahier des charges pouvant porter à confusion.

	\section{Compte-rendu}
	\paragraph{Légende la colonne Nature :}{
		\begin{itemize}
			\item I : Information. Toute affirmation échangée pendant la réunion, sans nécessairement donner lieu à débat.
			\item D : Décision. Toute affirmation qui a fait l'objet d'un débat et d'un accord, consensus de manière collégiale. 
		\end{itemize}
	}
	\begin{longtable}{|l|l|}
 	\hline
  	Nature & Texte \\ \hline
  	\hline
    I & \retourLigne{Les utilisateurs de l'application seront les organisateurs du BDE au sens large. Ceci n'est donc\\ pas réservé au seul Conseil d'Administration. Il y aura environ 60 utilisateurs.} \\ \hline
    D & \retourLigne{Les utilisateurs auront une connaissance basique de l'informatique. Ils seront formés pendant\\ environ 10 minutes.} \\ \hline
    D & \retourLigne{L'application sera disponible uniquement sur le système d'exploitation Android. Elle ne sera\\ pas distribuée sur le Play Store.} \\ \hline
    I & \retourLigne{L'application doit pouvoir récupérer les informations des adhérents du BDE\footnote{BDE : Bureau des Élèves} qui sont\\ déjà dans une base de données.} \\ \hline
    D & \retourLigne{L'inscription des utilisateurs se fera par un administrateur.} \\ \hline
    D & \retourLigne{Il n'est pas possible de créer son compte soi-même.} \\ \hline
    D & \retourLigne{Les utilisateurs doivent pouvoir voir les changements sur des documents sans avoir besoin\\ de recharger un élément.} \\ \hline
    D & \retourLigne{Les modifications sur un document n'ont pas besoin d'avoir un affichage en temps réel.} \\ \hline
    D & \retourLigne{Quand un autre utilisateur modifie actuellement le document que l'on est en train de consulter,\\ il faut l'indiquer.} \\ \hline
    I & \retourLigne{L'application doit respecter la charte graphique en vigueur du BDE.} \\ \hline
	\end{longtable}
\end{document}