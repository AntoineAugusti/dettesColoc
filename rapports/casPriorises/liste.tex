\documentclass[a4paper, 12pt, french]{article}
\usepackage[utf8]{inputenc}
\usepackage[french]{babel}
\usepackage[T1]{fontenc}
\usepackage[babel=true]{csquotes}
\usepackage{palatino}
\usepackage{hyperref}
\usepackage{eurosym}
\usepackage{graphicx}
\usepackage{array}
\usepackage{fancyhdr}
\usepackage{lastpage}
\usepackage{xspace}
\usepackage{longtable}
\usepackage{float}
\usepackage{geometry}
\addtolength{\oddsidemargin}{-.875in}
\addtolength{\evensidemargin}{-.875in}
\addtolength{\textwidth}{1.75in}

\addtolength{\topmargin}{-.5in}
\addtolength{\textheight}{1.75in}
%% -- Commandes personalisées

% Contrainte : chaîne de caractères sans espace, en minuscule, constituée seulement des 26 caractères non accentués de l’alphabet latin
\newcommand{\nomProjet}{enflatme\xspace}
% Contrainte : chaîne de caractères sans espace, en minuscule, constituée seulement des 26 caractères non accentués de l’alphabet latin
\newcommand{\nomEquipe}{teamflat\xspace}

\pagestyle{fancy}
% En-têtes
\renewcommand{\headrulewidth}{1pt}
\fancyhead[C]{\nomProjet} 
\fancyhead[L]{Compte-rendu de la réunion avec la MOA}
\fancyhead[R]{}

\renewcommand{\footrulewidth}{1pt}
\fancyfoot[C]{Version 1.0} 
\fancyfoot[L]{\nomEquipe\xspace G1}
\fancyfoot[R]{Page \thepage\xspace sur \pageref{LastPage}}

\newcommand{\retourLigne}[2][c]{\begin{tabular}[#1]{@{}l@{}}#2\end{tabular}}

%% -- Document
\begin{document}
	\begin{titlepage}
		\begin{center}
			\LARGE{\bsc{Liste des cas d'utilisation}} \\
		    \rule{\linewidth}{1.5pt}
		    \huge{\textbf{\nomProjet}}
		    \rule{\linewidth}{1.5pt} \newline{} \newline{}
		\end{center}
		\begin{center}
		    \large{Auteurs :}\\ Antoine \bsc{Augusti} (ANT)\\ Étienne \bsc{Batise} (BAT) \\ Thibaud \bsc{Dauce} (TIB)
		\end{center}
		\vspace{50px}
		\begin{center}
			\large{Date de publication :}\\ \today
		\end{center}
	\end{titlepage}
	\newpage

	\section{Ajouter Un membre}
	\begin{longtable}{|c|l|l|}
 	\hline
    1 & \retourLigne{ID}  & \retourLigne{obd01} \\ \hline
  	\hline
    2 & \retourLigne{Titre}  & \retourLigne{Ajouter un membre} \\ \hline
    3 & \retourLigne{Priorité}  & \retourLigne{1} \\ \hline
    4 & \retourLigne{Antécédents}  & \retourLigne{} \\ \hline
    5 & \retourLigne{Format}  & \retourLigne{A} \\ \hline
    6 & \retourLigne{Maquette}  & \retourLigne{1} \\ \hline
	\end{longtable}

	\section{Supprimer un membre}
	\begin{longtable}{|c|l|l|}
 	\hline
    1 & \retourLigne{ID}  & \retourLigne{obd02} \\ \hline
  	\hline
    2 & \retourLigne{Titre}  & \retourLigne{Supprimer un membre} \\ \hline
    3 & \retourLigne{Priorité}  & \retourLigne{1} \\ \hline
    4 & \retourLigne{Antécédents}  & \retourLigne{obd01} \\ \hline
    5 & \retourLigne{Format}  & \retourLigne{A} \\ \hline
    6 & \retourLigne{Maquette}  & \retourLigne{1} \\ \hline
	\end{longtable}


	\section{Connexion}
	\begin{longtable}{|c|l|l|}
 	\hline
    1 & \retourLigne{ID}  & \retourLigne{obd03} \\ \hline
  	\hline
    2 & \retourLigne{Titre}  & \retourLigne{Connexion} \\ \hline
    3 & \retourLigne{Priorité}  & \retourLigne{1} \\ \hline
    4 & \retourLigne{Antécédents}  & \retourLigne{obd01} \\ \hline
    5 & \retourLigne{Format}  & \retourLigne{A} \\ \hline
    6 & \retourLigne{Maquette}  & \retourLigne{1} \\ \hline
	\end{longtable}

\end{document}