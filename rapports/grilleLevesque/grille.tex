\documentclass[a4paper, 12pt, french]{article}
\usepackage[utf8]{inputenc}
\usepackage[french]{babel}
\usepackage[T1]{fontenc}
\usepackage[babel=true]{csquotes}
\usepackage{palatino}
\usepackage{hyperref}
\usepackage{eurosym}
\usepackage{graphicx}
\usepackage{array}
\usepackage{fancyhdr}
\usepackage{lastpage}
\usepackage{xspace}
\usepackage{longtable}
\usepackage{array}
\usepackage{pifont}
\usepackage{amssymb}
\usepackage{geometry}
\addtolength{\oddsidemargin}{-.875in}
\addtolength{\evensidemargin}{-.875in}
\addtolength{\textwidth}{1.75in}

\addtolength{\topmargin}{-.5in}
\addtolength{\textheight}{1.75in}

%% -- Commandes personalisées

% Contrainte : chaîne de caractères sans espace, en minuscule, constituée seulement des 26 caractères non accentués de l’alphabet latin
\newcommand{\nomProjet}{enflatme\xspace}
% Contrainte : chaîne de caractères sans espace, en minuscule, constituée seulement des 26 caractères non accentués de l’alphabet latin
\newcommand{\nomEquipe}{teamflat\xspace}
% Le type de document rédigé
\newcommand{\nomDocument}{Grille de Levesque}

\pagestyle{fancy}
% En-têtes
\renewcommand{\headrulewidth}{1pt}
\fancyhead[C]{\nomProjet} 
\fancyhead[L]{\nomDocument}
\fancyhead[R]{}

\renewcommand{\footrulewidth}{1pt}
\fancyfoot[C]{Version 1.0} 
\fancyfoot[L]{\nomEquipe\xspace G1}
\fancyfoot[R]{Page \thepage\xspace sur \pageref{LastPage}}

\renewcommand{\arraystretch}{1.1}

%% -- Document
\begin{document}
	\begin{titlepage}
		\begin{center}
			\LARGE{\bsc{Grille de Levesque}} \\
		    \rule{\linewidth}{1.5pt}
		    \huge{\textbf{\nomProjet}}
		    \rule{\linewidth}{1.5pt} \newline{} \newline{}
		\end{center}
		\begin{center}
		    \large{Auteurs :}\\ Antoine \bsc{Augusti} (ANT)\\ Étienne \bsc{Batise} (BAT) \\ Thibaud \bsc{Dauce} (TIB)
		\end{center}
		\vspace{50px}
		\begin{center}
			\large{Date de publication :}\\ \today
		\end{center}
	\end{titlepage}

	% Début du document
  \section{Contexte}
  L'objectif de ce document est de présenter la grille de Levesque de notre projet. Celle-ci fait suite à la séance de remue-méninge que nous avons effectuée le 14 février 2014 en compagnie de l'équipe bidideco. Au cours de cette séance, nous avons abordé énormément de besoins et d'opportunités que pourraient amener notre projet. Ainsi nous avons dû modifier notre liste  d'idées brutes et rajouter quelques idées pour établir une grille cohérente.

	\section{Grille de Levesque}
	L'objectif principal retenu est d'\textbf{améliorer la vie des colocataires}.
  Les idées non présentes dans la liste des idées brutes possèdent un identifiant alphabétique, celles possédant un identifiant numérique reprennent l'identifiant de la liste des idées brutes du remue-méninge.

  \scriptsize
	\begin{longtable}{|l|c|c|c|c|c|c|c|}
 	\hline
  	ID de l'idée & Fait / Donnée / Observation & Symptôme & Problème & Besoin & Objectif & Opportunité & Solution \\ \hline
  	\hline
    1 &  Gestion des dettes & & & \checkmark & & & \\ \hline
    2 &  Suivi des tâches ménagères & & & & & & B \\ \hline
    3 &  Plan de frigidaire & & & \checkmark & & & \\ \hline
    4 &  Paiement du loyer & & & \checkmark & & & \\ \hline
    5 &  Liste des courses & & & & & & D\\ \hline
    6 &  Calendrier & & & & & \checkmark & \\ \hline
    7 &  Vaisselle & & & & & & B\\ \hline
    8 &  Facture (consommation eau et électricité) & & & \checkmark & & & \\ \hline
    9 &  Gestion des lessives & & & & & & E\\ \hline
    11 & Gestion des voitures & & & C & & & \\ \hline
    12 & Changement de colocation (personne) & & & & & \checkmark & \\ \hline
    13 & Changement de colocation (adresse) & & & & & \checkmark & \\ \hline
    14 & Règlements / Commandements & & & \checkmark & & & \\ \hline
    16 & Gestion des clés & & & \checkmark & & & \\ \hline
    17 & Partage de fichiers & & & & & \checkmark & \\ \hline
    18 & Gestion du courrier & & & \checkmark & & & \\ \hline
    19 & Gestion des poubelles & & & \checkmark & & & \\ \hline
    20 & Heure de repos / de SDB\footnote{SDB : salle de bain.} / de baise & & & E & & & \checkmark \\ \hline
    21 & Bruits tardifs & E & & & & & \\ \hline
    22 & Organisation de soirées & & & \checkmark & & & \\ \hline
    23 & Liste noire (personnes interdites) & & & \checkmark & & & \\ \hline
    24 & Mise en place de statistiques & & & & & \checkmark & \\ \hline
    25 & Citations cultes & & & & & \checkmark & \\ \hline
    26 & Jeux entre colocataires & & & & & \checkmark & \\ \hline
    27 & \textit{Achievements} & & & & & \checkmark & \\ \hline
    28 & Stock de capotes / de pilules & & & \checkmark & & & \\ \hline
    29 & Communications inter-colocations & & & & & \checkmark & \\ \hline
    30 & Défis inter-colocations & & & & & \checkmark & \\ \hline
    31 & \textit{Couch-surfing} & & & & & \checkmark & \\ \hline
    32 & Objets trouvés / perdus & & & & & & C \\ \hline
    33 & Gestion des meubles achetés & & & C & & & \\ \hline
    35 & Kikifélabouffe \texttrademark& & & B & & & \\ \hline
    36 & Gestion de l'alcool& & & & & & D\\ \hline
    37 & Gestion du son& & & E & & & \\ \hline
    38 & Alertes générales (visites impromptues)& & & \checkmark & & & \\ \hline
    39 & Numéros d'urgence& & & \checkmark & & & \\ \hline
    40 & Carnet d'adresses& & & & & \checkmark & \\ \hline
    41 & Carnet de santé& & & \checkmark & & & \\ \hline
    42 & Sélection de films& & & E & & & \\ \hline
    A &  \textbf{Améliorer la vie des colocataires} & & & & \checkmark & & \\ \hline
    B &  Les tâches sont mal réparties & & \checkmark & & & & \\ \hline
    C &  Chacun possède ses biens & & \checkmark & & & & \\ \hline
    D &  Il manque parfois des consommables & & \checkmark & & & & \\ \hline
    E & Les loisirs des uns embêtent les autres & & \checkmark & & & & \\ \hline
	\end{longtable}
\end{document}