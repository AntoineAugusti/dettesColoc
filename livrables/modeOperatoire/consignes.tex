\documentclass[a4paper, 12pt, french]{article}
\usepackage[utf8]{inputenc}
\usepackage[french]{babel}
\usepackage[T1]{fontenc}
\usepackage[babel=true]{csquotes}
\usepackage{hyperref}
\usepackage{eurosym}
\usepackage{graphicx}
\usepackage{array}
\usepackage{fancyhdr}
\usepackage{lastpage}
\usepackage{xspace}
\usepackage{longtable}
\usepackage{float}
\usepackage{geometry}
\addtolength{\oddsidemargin}{-.875in}
\addtolength{\evensidemargin}{-.875in}
\addtolength{\textwidth}{1.75in}

\addtolength{\topmargin}{-.5in}
\addtolength{\textheight}{1.75in}
%% -- Commandes personalisées

% Contrainte : chaîne de caractères sans espace, en minuscule, constituée seulement des 26 caractères non accentués de l’alphabet latin
\newcommand{\nomProjet}{enflatme\xspace}
% Contrainte : chaîne de caractères sans espace, en minuscule, constituée seulement des 26 caractères non accentués de l’alphabet latin
\newcommand{\nomEquipe}{teamflat\xspace}

\pagestyle{fancy}
% En-têtes
\renewcommand{\headrulewidth}{1pt}
\fancyhead[C]{\nomProjet} 
\fancyhead[L]{Mode opératoire}
\fancyhead[R]{}

\renewcommand{\footrulewidth}{1pt}
\fancyfoot[C]{Version 1.0} 
\fancyfoot[L]{\nomEquipe\xspace G1}
\fancyfoot[R]{Page \thepage\xspace sur \pageref{LastPage}}

\newcommand{\retourLigne}[2][c]{\begin{tabular}[#1]{@{}l@{}}#2\end{tabular}}

%% -- Document
\begin{document}
	\begin{center}
	    \rule{\linewidth}{1pt}
		\LARGE{\bsc{Mode opératoire}}
	    \rule{\linewidth}{1pt} \newline{} \newline{}
	\end{center}
	\begin{center}
	    \large{Auteurs :}\\ Antoine \bsc{Augusti} (ANT)\\ Étienne \bsc{Batise} (BAT) \\ Thibaud \bsc{Dauce} (TIB)
	\end{center}
	\vspace{20px}
	\begin{center}
		\large{Date de publication :}\\ \today
	\end{center}

	\section{Préliminaires}
	Il est nécessaire d'effectuer les étapes ci-dessous pour faire fonctionner l'application web et la base de données.
	\begin{enumerate}
		\item Récupérer notre archive de livrables \texttt{v1.0} à l'adresse \url{https://github.com/teamflat/enflatme/releases}
		\item Dézipper cette archive
	\end{enumerate}

	\section{Lancement de l'application web en local}
	\begin{enumerate}
		\item Ouvrir un terminal et se rendre dans le dossier où l'archive a été dézippée
		\item Se rendre dans le dossier \texttt{livrables/obde-gwt}
		\item Lancer la commande \texttt{mvn gwt:run}
		\item Ouvrir un navigateur web et se rendre à l'adresse \url{http://127.0.0.1:8888/Obde.html?gwt.codesvr=127.0.0.1:9997}
		\item Si le plugin GWT est manquant dans le navigateur, l'installer.
	\end{enumerate}

	\section{Déploiement de l'application web}
	\begin{enumerate}
		\item Ouvrir un terminal et se rendre dans le dossier où l'archive a été dézippée
		\item Se rendre dans le dossier \texttt{livrables/obde-gwt}
		\item Lancer la commande \texttt{mvn gwt:redeploy}
		\item L'application est disponible à l'adresse \url{http://casisbelli-dev.insa-rouen.fr/obde/}
	\end{enumerate}

	\section{Création de la base de données}
	\begin{enumerate}
		\item Ouvrir un terminal et se rendre dans le dossier où l'archive a été dézippée
		\item Se rendre dans le dossier \texttt{livrables/requetesSQL}
		\item Importer les fichiers SQL dans l'ordre suivant : \texttt{teamflat.ddl.sql} puis \texttt{teamflat.dml.sql}. Ceci peut être fait en ligne de commandes ou à l'aide de phpMyAdmin. La commande pour importer un fichier SQL dans une base de données MySQL est la suivante : \texttt{mysql -u UTILISATEUR -p -h localhost NOM-BASE < fichier.sql}.
	\end{enumerate}

\end{document}