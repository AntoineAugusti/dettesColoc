\documentclass[a4paper, 12pt, french]{article}
\usepackage[utf8]{inputenc}
\usepackage[french]{babel}
\usepackage[T1]{fontenc}
\usepackage[babel=true]{csquotes}
\usepackage{palatino}
\usepackage{hyperref}
\usepackage{eurosym}
\usepackage{graphicx}
\usepackage{array}
\usepackage{fancyhdr}
\usepackage{lastpage}
\usepackage{xspace}
\usepackage{longtable}
\usepackage{float}
\usepackage{geometry}
\addtolength{\oddsidemargin}{-.875in}
\addtolength{\evensidemargin}{-.875in}
\addtolength{\textwidth}{1.75in}

\addtolength{\topmargin}{-.5in}
\addtolength{\textheight}{1.75in}
%% -- Commandes personalisées

% Contrainte : chaîne de caractères sans espace, en minuscule, constituée seulement des 26 caractères non accentués de l’alphabet latin
\newcommand{\nomProjet}{enflatme\xspace}
% Contrainte : chaîne de caractères sans espace, en minuscule, constituée seulement des 26 caractères non accentués de l’alphabet latin
\newcommand{\nomEquipe}{teamflat\xspace}

\pagestyle{fancy}
% En-têtes
\renewcommand{\headrulewidth}{1pt}
\fancyhead[C]{\nomProjet} 
\fancyhead[L]{Liste des livrables}
\fancyhead[R]{}

\renewcommand{\footrulewidth}{1pt}
\fancyfoot[C]{Version 1.0} 
\fancyfoot[L]{\nomEquipe\xspace G1}
\fancyfoot[R]{Page \thepage\xspace sur \pageref{LastPage}}

\newcommand{\retourLigne}[2][c]{\begin{tabular}[#1]{@{}l@{}}#2\end{tabular}}

%% -- Document
\begin{document}
	\begin{center}
	    \rule{\linewidth}{1pt}
		\LARGE{\bsc{Liste des livrables}}
	    \rule{\linewidth}{1pt} \newline{} \newline{}
	\end{center}
	\begin{center}
	    \large{Auteurs :}\\ Antoine \bsc{Augusti} (ANT)\\ Étienne \bsc{Batise} (BAT) \\ Thibaud \bsc{Dauce} (TIB)
	\end{center}
	\vspace{20px}
	\begin{center}
		\large{Date de publication :}\\ \today
	\end{center}

	\begin{longtable}{|c|c|}
 	\hline
 	Vignette & Document \\ \hline
 	\hline
	1 & Liste d'idées \\ \hline
	2 & Grille de Levesque  \\ \hline
	3-8 & Cahier des charges \\ \hline
	9R & Liste de questions \\ \hline
	9V & Liste de pièces à conviction \\ \hline
	9I & Pièces à conviction \\ \hline
	10 & Liste de cas priorisés \\ \hline
	11 & Cas documentés - Cockburn \\ \hline
	12 & Cas documentés - Diagrammes d'activité \\ \hline
	13 & Cas documentés - Diagramme de séquence système \\ \hline
	14 & Cas documentés - Maquettes \\ \hline
	16 & Modèle d’usage \\ \hline
	18 & Diagramme d’objets \\ \hline
	19 & Diagramme de classes participantes \\ \hline
	20 & Diagramme de séquence Jacobson d’analyse \\ \hline
	21 & Modèle du domaine \\ \hline
	22 & Mode opératoire \\ \hline
	23 & Guide utilisateur \\ \hline
	24 & Exigences qualifiées \\ \hline
	\end{longtable}

\end{document}