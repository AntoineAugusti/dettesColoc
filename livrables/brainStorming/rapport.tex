\documentclass[a4paper, 12pt, french]{article}
\usepackage[utf8]{inputenc}
\usepackage[french]{babel}
\usepackage[T1]{fontenc}
\usepackage[babel=true]{csquotes}
\usepackage{palatino}
\usepackage{hyperref}
\usepackage{eurosym}
\usepackage{graphicx}
\usepackage{array}
\usepackage{fancyhdr}
\usepackage{lastpage}
\usepackage{xspace}
\usepackage{longtable}
\usepackage{array}

%% -- Commandes personalisées

% Contrainte : chaîne de caractères sans espace, en minuscule, constituée seulement des 26 caractères non accentués de l’alphabet latin
\newcommand{\nomProjet}{enflatme\xspace}
% Contrainte : chaîne de caractères sans espace, en minuscule, constituée seulement des 26 caractères non accentués de l’alphabet latin
\newcommand{\nomEquipe}{teamflat\xspace}
% Le type de document rédigé
\newcommand{\nomDocument}{Remue-méninge}

\pagestyle{fancy}
% En-têtes
\renewcommand{\headrulewidth}{1pt}
\fancyhead[C]{\nomProjet} 
\fancyhead[L]{\nomDocument}
\fancyhead[R]{}

\renewcommand{\footrulewidth}{1pt}
\fancyfoot[C]{Version 1.0} 
\fancyfoot[L]{\nomEquipe\xspace G1}
\fancyfoot[R]{Page \thepage\xspace sur \pageref{LastPage}}

\renewcommand{\arraystretch}{1.2}

%% -- Document
\begin{document}
	\begin{titlepage}
		\begin{center}
			\LARGE{\bsc{Remue-méninge}} \\
		    \rule{\linewidth}{1.5pt}
		    \huge{\textbf{\nomProjet}}
		    \rule{\linewidth}{1.5pt} \newline{} \newline{}
		\end{center}
		\begin{center}
		    \large{Auteurs :}\\ Antoine \bsc{Augusti} (ANT)\\ Étienne \bsc{Batise} (BAT) \\ Thibaud \bsc{Dauce} (TIB)
		\end{center}
		\vspace{50px}
		\begin{center}
			\large{Date de publication :}\\ \today
		\end{center}
	\end{titlepage}

	% Début du document
	\section{Contexte}
	Cette séance de remue-méninge s'est déroulée le vendredi 14 mars 2014 avec les équipes teamflat et bidideco.

	\section{Liste d'idées}
	Retrouvez ci-dessous la liste des idées brutes que nous avons abordé durant cette séance.

	\begin{longtable}{|l|c|}
 	\hline
  	ID de l'idée & Texte de l'idée \\ \hline
  	\hline
  	1 &  Gestion des dettes \\ \hline
  	2 &  Suivi des tâches ménagères \\ \hline
  	3 &  Plan de frigidaire \\ \hline
  	4 &  Loyer \\ \hline
  	5 &  Liste des courses \\ \hline
  	6 &  Calendrier \\ \hline
  	7 &  Vaisselle \\ \hline
  	8 &  Facture (consommation eau et électricité) \\ \hline
  	9 &  Gestion des lessives \\ \hline
  	10 &  Nettoyage de vomi \\ \hline
  	11 & Gestion des voitures \\ \hline
  	12 & Changement de colocation (personne) \\ \hline
  	13 & Changement de colocation (adresse) \\ \hline
  	14 & Règlements / Commandements \\ \hline
  	15 & Pourcentage de peau visible en fonction du sexe \\ \hline
  	16 & Gestion des clés \\ \hline
  	17 & Partage de fichiers \\ \hline
  	18 & Gestion du courrier \\ \hline
  	19 & Gestion des poubelles \\ \hline
  	20 & Heure de repos / de salle de bain / de baise \\ \hline
  	21 & Couvre-feu / couvre-chef \\ \hline
  	22 & Organisation de soirées \\ \hline
  	23 & Liste noire (personnes interdites) \\ \hline
  	24 & Mise en place de statistiques \\ \hline
  	25 & Citations cultes \\ \hline
  	26 & Jeux entre colocataires \\ \hline
  	27 & \textit{Achievements} \\ \hline
  	28 & Stock de capotes / de pilules du lendemain \\ \hline
  	29 & Communications inter-colocations \\ \hline
  	30 & Défis inter-colocations \\ \hline
  	31 & \textit{Couch-surfing} \\ \hline
  	32 & Objets trouvés / perdus \\ \hline
  	33 & Gestion des meubles achetés \\ \hline
  	34 & Gestion des gestionnaires \\ \hline
  	35 & Kikifélabouffe \texttrademark\\ \hline
  	36 & Gestion de l'alcool\\ \hline
  	37 & Gestion du son\\ \hline
  	38 & Alertes générales (visites impromptues)\\ \hline
  	39 & Numéros d'urgence\\ \hline
  	40 & Carnet d'adresses\\ \hline
  	41 & Carnet de santé\\ \hline
  	42 & Sélection de films\\ \hline
	\end{longtable}
\end{document}