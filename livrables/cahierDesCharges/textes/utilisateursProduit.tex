\subsection{Utilisateurs directs du produit}
Les utilisateurs de notre application seront les personnes vivant dans une colocation. La majorité des personnes vivant dans une colocation sont des étudiants puisque ceux-ci ont souvent peu de moyens financiers et choisissent de vivre en colocation pour réaliser des économies.\\

Notre cible est donc un public jeune, entre 18 et 25 ans, autant hommes que femmes. Certains jeunes actifs font également le choix de la colocation. Notre public utilise très fréquemment et maîtrise les dernières nouveautés technologiques et télécharge très régulièrement de nouvelles applications.\\

Concernant les centres d'intérêt de notre cible ils sont autour des sorties, de la musique, des soirées, des jeux-vidéos, des sports\dots Notre cible n'a pas beaucoup de temps à consacrer à la réalisation de tâches ménagères et n'apprécie pas perdre du temps avec ceci. 

\subsection{Priorité assignée aux utilisateurs}
Les utilisateurs sont notre priorité ultime car nous réalisons une application pour répondre à leurs besoins. De plus, nous comptons sur nos utilisateurs pour réaliser la promotion de notre application et ainsi faire en sorte que celle-ci ait une croissance virale.

\subsection{Implication nécessaire de la part des utilisateurs dans le projet}
Aucune implication particulière des utilisateurs n'est nécessaire pour mener à bien le projet. Toutefois, étant donné qu'il est primordial de concevoir un produit qui convient à notre cible que sont les colocataires, il est important de travailler de manière proche avec certains d'entre eux afin d'avoir leur avis sur l'avancement et l'orientation du projet.

\subsection{Utilisateurs concernés par les opérations de maintenance du produit}
Aucun utilisateur ne devra être concerné par des opérations de maintenance car les données devront être stockée sur un des serveurs de la société Teamflat.